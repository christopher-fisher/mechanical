\documentclass{article}
\usepackage[utf8]{inputenc}
\def\code#1{\texttt{#1}}

\title{Mechanical Roadmap v1.0.1}
\author{Chris Fisher
\\https://github.com/christopher-fisher/mechanical}
\date{October 9 2020}

\begin{document}

\maketitle

\section{What is Mechanical?}
Mechanical is a toolkit for the rapid prototyping of common types of generative machine learning projects. It is not intended to replace the full TensorFlow library, but rather to serve as a more approachable entry point to the sorts of projects that are possible with TensorFlow.

%\section{Philosophy}

\section{Goals}
\begin{itemize}
    \item Make machine learning projects more approachable
    \item Create a simple yet highly configurable tool for rapid prototyping of machine learning projects
    \item Learn more about TensorFlow in the process of building the Mechanical toolkit
\end{itemize}
\subsection{Long Term Goals}
\begin{itemize}
    \item Eventually implement a web front end using Flask and React in order to better learn those frameworks
\end{itemize}

\section{Development Milestones}
\subsection{v0.1: Text Generation}
Projected release: October 17 2020
\begin{itemize}
    \item Implement and adapt the text generation model I implemented in my project Pynk Floyd
    \item All functionality and model parameters adjusted via \code{textgen\_config.py}
    \item Currently the toolkit is utilized by importing it into a separate script file. This will be a temporary measure until a CLI is implemented for v0.2
\end{itemize}
\subsection{v0.2: Command Line Interface}
Projected release: TBD following release of v0.1
\begin{itemize}
    \item Implement command line interface for toolkit
    \item Allow for user-provided custom configuration files
\end{itemize}
\subsection{More features TBD}
\end{document}
